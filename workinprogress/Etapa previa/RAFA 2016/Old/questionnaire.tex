\documentclass[
  % Babel language, also used to load translations
  %english,
  spanish,
  % Use A4 paper size, you can change this to eg. letterpaper if you need
  % the letter format. The normal methods to modify the paper size should
  % be picked up by SDAPS automatically.
  a4paper, % setting this might break the example scan unfortunately
  % letterpaper
  %
  % If you need it, you can add a custom barcode at the center
  globalid=1003, % Piloto facu
  %
  % And the following adds a per sheet barcode at the bottom left
  print_questionnaire_id,
  %
  % You can choose between twoside and oneside. twoside is the default, and
  % requires the document to be printed and scanned in duplex mode.
  % oneside,
  %
  % The following options make sense so that we can get a better feel for the
  % final look.
  pagemark,
  stamp]{sdaps}
\usepackage[utf8]{inputenc}
% For demonstration purposes
\usepackage{multicol}

\author{}
\title{}

\begin{document}
  % Everything you do should be done inside the questionnaire environment.

  % If you don't like the default text at the beginning of each questionnaire
  % you can remove it with the optional [noinfo] parameter for the environment 
  \begin{questionnaire}[noinfo]
    % There is a predefined "info" style to hilight some text.
    %\begin{info}
    %  Some information here. Nothing special, just adds a line above/below.
    %\end{info}


    \section{Información personal}

	\begin{choicequestion}[3]{Seleccione su edad}
	\choiceitem{Menos de 5 años}
	\choicemulticolitem{2}{Entre 5 y 12 años}
	\choiceitem{Entre 13 y 17}
	\choicemulticolitem{2}{Entre 18 y 24 años}
	\choiceitem{Entre 25 y 40 años}
	\choicemulticolitem{2}{Más de 40 años}

	\end{choicequestion}

	\begin{choicequestion}[3]{Seleccione su área de interés profesional}
	\choiceitem{Psicología}
	\choicemulticolitem{2}{Biología/Química/Geología/Paleontología}
	\choiceitem{Artes plásticas}
	\choicemulticolitem{2}{Ciencias Médicas (medicina, kinesiología, odontología, etc)}

	\choiceitem{Arquitectura/Diseño}
	\choicemulticolitem{2}{Física/Cs Atmósfera/Oceanografía}
	\choiceitem{Música}
	\choicemulticolitem{2}{Turismo}
	
	\choiceitem{Derecho}
	\choicemulticolitem{2}{Matemática/Computación/Ing. Sistemas}
	\choiceitem{Farmacia/Bioquímica}
	\choicemulticolitem{2}{Profesorado/Docencia(Cs Naturales y afines)}

	\choiceitem{Letras}
	\choicemulticolitem{2}{Profesorado/Docencia(Cs Sociales, letras, artes, afines)}
	\choiceitem{Deportes/Danzas}
	\choicemulticolitem{2}{Ingeniería (no sistemas)}

	\choiceitem{Cs Sociales}
	\choicemulticolitem{2}{Gastronomía}
	\choiceitem{Marketing/Publicidad}
	\choicemulticolitem{2}{Cs Económicas (economía, contador, etc)}

	\choiceitem{Agronomía/Veterinaria}

	\choiceitemtext{0.5cm}{1}{Otra opción:}

	\end{choicequestion}

	\begin{choicequestion}[3]{Seleccione su nivel de estudios formales}

	\choiceitem{Primaria Completa}
	\choicemulticolitem{2}{Secundaria Completa}
	\choiceitem{Terciario Completo}
	\choicemulticolitem{2}{Universidad Completa}
	\choiceitem{Universidad en curso}
	\choicemulticolitem{2}{Posgrado Completo}


	\end{choicequestion}

    \section{Acerca de lo que acaba de ver evalue la factibilidad de cada una de las siguientes opciones}
    
      \begin{multicols}{2}
      \begin{choicegroup}{Marque una sola opción en la primera columna, y todas las demás que le parezcan posibles en la segunda.}
	% We have to add the possible choices at the start.
      \groupaddchoice{Mi elección}
      \groupaddchoice{Podría ser}
      
      % After that it is possible to add each question.
      \choiceline{Calor}
      \choiceline{Telequinesis}
      \choiceline{Magnetismo}
      \choiceline{Energía Cósmica}
      \choiceline{Física Cuántica}
      \choiceline{Presión}
      \choiceline{Aura Corporal}
      \choiceline{Electricidad}
      \choiceline{Poderes Mágicos}
      \choiceline{No lo se}
      \end{choicegroup}

	\setcounter{markcheckboxcount}{7}
	\begin{markgroup}{Marque cuan factible le parece cada opción, la haya seleccionado o no.}
	  \markline{}{Mucho}{Nada}
	  \markline{}{Mucho}{Nada}
	  \markline{}{Mucho}{Nada}  
	  \markline{}{Mucho}{Nada}  
	  \markline{}{Mucho}{Nada}
	  \markline{}{Mucho}{Nada}
	  \markline{}{Mucho}{Nada}  
	  \markline{}{Mucho}{Nada}  
	  \markline{}{Mucho}{Nada}  
	  \markline{}{Mucho}{Nada}
	\end{markgroup}

    \end{multicols}

    \textbox{1cm}{Algo que quiera agregar...}
    \setcounter{markcheckboxcount}{10}  
    \singlemark{¿Cuan relevante le parece en el proceso cómo se acerca la mano?}{Irrelevante}{Determinante}


    % Use \addinfo to add metadata (which is printed on the report later on)
    % \addinfo{Date}{\today}

    \newpage

    \section{Nuevamente, acerca de lo que acaba de ver, evalue la factibilidad de cada una de las siguientes opciones}
    
      \begin{multicols}{2}
      \begin{choicegroup}{Marque una sola opción en la primera columna, y todas las demás que le parezcan posibles en la segunda.}
	% We have to add the possible choices at the start.
      \groupaddchoice{Mi elección}
      \groupaddchoice{Podría ser}
      
      % After that it is possible to add each question.
      \choiceline{Calor}
      \choiceline{Telequinesis}
      \choiceline{Magnetismo}
      \choiceline{Energía Cósmica}
      \choiceline{Física Cuántica}
      \choiceline{Presión}
      \choiceline{Aura Corporal}
      \choiceline{Electricidad}
      \choiceline{Poderes Mágicos}
      \choiceline{No lo se}
      \end{choicegroup}

	\setcounter{markcheckboxcount}{7}
	\begin{markgroup}{Marque cuan factible le parece cada opción, la haya seleccionado o no.}
	  \markline{}{Mucho}{Nada}
	  \markline{}{Mucho}{Nada}
	  \markline{}{Mucho}{Nada}  
	  \markline{}{Mucho}{Nada}  
	  \markline{}{Mucho}{Nada}
	  \markline{}{Mucho}{Nada}
	  \markline{}{Mucho}{Nada}  
	  \markline{}{Mucho}{Nada}  
	  \markline{}{Mucho}{Nada}  
	  \markline{}{Mucho}{Nada}
	\end{markgroup}

    \end{multicols}

    \begin{choicequestion}[3]{¿Piensa que hay algún truco que no notó y explica lo ocurrido?}
      \choiceitem{Si}
      \choiceitem{No}
      \choiceitem{No se}
    \end{choicequestion}

	\section{Un poco más de información personal...}

    \begin{choicegroup}{¿Cuán a gusto se siente realizando las siguientes tareas?}
      % We have to add the possible choices at the start.
      \groupaddchoice{Me encanta}
      \groupaddchoice{Me gusta}
      \groupaddchoice{Ni}
      \groupaddchoice{No me gusta}
      \groupaddchoice{Lo odio}

      % After that it is possible to add each question.
      \choiceline{Bailar}
      \choiceline{Cocinar}
      \choiceline{Hacer ejercicios de matemática}
      \choiceline{Mirar películas/series}
      \choiceline{Construir cosas}
      \choiceline{Trabajo de oficina}
      \choiceline{Tarot}
      \choiceline{Leer libros}
      \choiceline{Prácticas religiosas}
      \choiceline{Reparar cosas}
      \choiceline{Programar}
      \choiceline{Correr}
      \choiceline{Resolver Acertijos}
      \choiceline{Astrología}
      \choiceline{Jugar deportes en equipo}
      \choiceline{Debatir}
      \choiceline{Trucos de magia}
      \choiceline{Electrónica}


    \end{choicegroup}


  \end{questionnaire}
\end{document}

